\documentclass[a4paper, 11pt, twocolumn]{article}
\usepackage[left=1.5cm,text={18cm, 25cm},top=2.5cm]{geometry}
\usepackage[czech]{babel}
\usepackage[utf8]{inputenc}
\usepackage[IL2]{fontenc}
\usepackage{times}
\usepackage{amsmath, amssymb, amsthm, amsfonts, array}
\usepackage[unicode]{hyperref}

\newtheorem{definice}{Definice}
\newtheorem{sentence}{Věta}

\begin{document}
	\begin{titlepage}
		\begin{center}
			{\Huge\textsc{
				Fakulta informačních technologií \\ \vspace{0.4em}
				Vysoké učení technické v~Brně
			}}
			\vspace{\stretch{0.382}}
			{\LARGE
				\\ Typografie a~publikování -- 2. projekt \\
				\vspace{0.3em}
				Sazba dokumentů a~matematických výrazů
			}
			\vspace{\stretch{0.618}}
		\end{center}

		{\Large
			\the\year
			\hfill
			Filip Pospíšil (xpospi0f)
		}
  \end{titlepage}

	\section*{Úvod}
	V~této úloze si vyzkoušíme sazbu titulní strany, matematic\-kých vzorců, prostředí a~dalších textových struktur obvyklých pro~technicky zaměřené texty (například rovnice \eqref{eq:1} nebo Definice \ref{def:1} na~straně \pageref{def:1}). Pro odkazovaní na~vzorce a~struktury zásadně používáme příkaz \verb|\label| a \verb|\ref| případně \verb|\pageref| pokud se chceme odkázat na~stranu výskytu.
	\par
	Na titulní straně je využito sázení nadpisu podle optického středu s~využitím zlatého řezu. Tento postup byl probírán na~přednášce. Dále je použito odřádkování se zadanou relativní velikostí 0.4\,em~a 0.3\,em.

	\section{Matematický text}
	Nejprve se podíváme na sázení matematických symbolů a~výrazů v~plynulém textu včetně sazby definic a~vět s~využitím balíku \texttt{amsthm}. Rovněž použijeme poznámku pod čarou s~použitím příkazu \verb|\footnote|. Někdy je vhodné použít konstrukci \verb|\mbox{}|, která říká, že text nemá být zalomen.

    \begin{definice}
			\label{def:1}
			\emph{Zásobníkový automat} (ZA) je definován jako sedmice tvaru $A = (Q, \Sigma, \Gamma, \delta, q_0, Z_0, F)$, kde:
			\begin{itemize}%
					\item $Q$ je konečná množina \emph{vnitřních (řídicích) stavů},
					\item $\Sigma$ je konečná \emph{vstupní abeceda},
					\item $\Gamma$ je konečná \emph{zásobníková abeceda},
					\item $\delta$ je \emph{přechodová funkce} $Q \times (\Sigma \cup \{\epsilon \}) \times \Gamma \rightarrow 2^{Q \times \Gamma^\ast}$,
					\item $q_0 \in Q $ je \emph{počáteční stav}, $Z_0 \in \Gamma$ je \emph{startovací symbol zásobníku} a $F \subseteq Q$ je množina \emph{koncových stavů}.%
			\end{itemize}

			\emph{Nechť $P = (Q, \Sigma, \Gamma, \delta, q_0, Z_0, F)$ je zásobníkový automat}. Konfigurací \emph{nazveme trojici $(q, w, \alpha) \in Q \times \Sigma^\ast \times \Gamma^*$, kde $q$ je aktuální stav vnitřního řízení, $w$ je dosud nezpracovaná část vstupního řetězce a $\alpha = Z_{i_1} Z_{i_2}\dots Z_{i_k}$ je obsah zásobníku}\footnote{$Z_{i_1}$ je vrchol zásobníku}.
		\end{definice}

    \subsection{Podsekce obsahující větu a odkaz}
    \begin{definice}
			\label{def:2}
			\emph{Řetězec $w$ nad abecedou $\Sigma$ je přijat ZA} $A$~jestliže $(q_0, w, Z_0) \underset{A}{\overset{\ast}{\vdash}} (q_F, \epsilon, \gamma)$ pro nějaké $\gamma \in \Gamma^\ast$ a $q_F \in F$. Množinu\,$L(A) = \{w\;|\;w$ je přijat ZA $ A \} \subseteq \Sigma^\ast$\,nazýváme \emph{jazyk přijímaný TS $M$.}%
        \end{definice}
		Nyní si vyzkoušíme sazbu vět a důkazů opět s~použítím balíku \verb|amsthm|.

    \begin{sentence}
        Třída jazyků, které jsou přijímány ZA, odpovídá \emph{bezkontextovým jazykům}.
    \end{sentence}
    \begin{proof}
        V~důkaze vyjdeme z~Definice 1 a 2.
    \end{proof}
  \section{Rovnice a odkazy}
    Složitější matematické formulace sázíme mimo plynulý text. Lze umístit několik výrazů na jeden řdek, ale pak je třeby tyto vhodně oddělit, například příkazem \verb|\quad|.%

		\[ \sqrt[i]{x_i^3}\text{\quad kde } x_i \text{ je } i\text{-té sudé číslo splňující\quad}x_i^{2-x_i^{i^2}}\leq x_i^{y_i^{3}}\]
		\par
    V~rovnici \eqref{eq:1} jsou využity tři typy závorek s~různou explicitně definovanou velikostí.
		\begin{eqnarray}
			 \label{eq:1}
			 x & = & \bigg[ \Big\{\big[a + b\big] * c\Big\}^d \ominus 1 \bigg]^{(1/2)}\\
			 y & = & \lim_{x\to\infty} \frac{\frac{1}{\log_{10} x}}{\sin^2x + \cos^2x
			 \nonumber}
    \end{eqnarray}
		V~této větě vidíme, jak vypadá impicitní vyázení limity $\lim_{x\to\infty} f(n)$ v~normálním odstavci textu. Podobně je to i s~dalšími symboly jako $\Pi^{n}_{i=1} 2^{i}$ či $\cap_{A\in \mathcal{B}}\ ^{A.}$ V~případě vzorců $\lim\limits_{x\to\infty}$ a $\Pi^{n}_{i=1} 2^{i}$ jsme si vynutili méně úspornou sazbu příkazem \verb|\limits|.

    \begin{eqnarray}
		\int^b_a g(x) \, \mathrm{d}x & = & - \int\limits^b_a f(x) \, \mathrm{d}x \\
		\overline{\overline{A \wedge B}} &\Leftrightarrow& \overline{\overline{A} \vee \overline{B}}
    \end{eqnarray}

    \section{Matice}
    Pro sázení matic se velmi často používá prostředí \verb|array| a závorky (\verb|\left|, \verb|\right|).
		\[\left[\begin{array}{ccc}%
									 &   \widehat{\beta + \gamma} & \hat{\pi}    \\
					 \vec{a} &   \overset{\longleftrightarrow}{AC} &
			 \end{array}\right] = 1 \Longleftrightarrow \mathbb{Q} = \mathbf{R}\]%

\[\mathbf{A} =\left| \begin{array}{cccc}
					 a_{11}  & a_{12} & \dots  & a_{1n} \\
					 a_{21}  & a_{22} & \dots  & a_{2n} \\
					 \vdots  & \vdots & \ddots & \vdots \\
					 a_{m1} & a_{m2} & \dots  & a_{mn}
			 \end{array}\right| =
			 \begin{array}{cc}
		\,t & u    \\
		v & \:w
\end{array} = tw - uv\]

	Prostředí \texttt{array} lze úspěšně využít i jinde.
	\[
		{\binom{n}{k}} =
    \left\{\begin{array}{ll}
        0                   & \mbox{pro } k\:< 0 \mbox{ nebo } k\: > n \\
        \frac{n!}{k!(n-k)!} & \mbox{pro } 0 \leq k\: \leq n
    \end{array}\right.
\]%
\end{document}
